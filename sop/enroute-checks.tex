\section{EN ROUTE CHECKS \& SCANS}

\subsection{Checks}

\subsubsection{Ops Check}

An Ops check is a quick check over the aircraft that puts several reminders in
the front of the mind, notably fuel and switchology. It is most useful for
verifying quickly switches you don't often visually check haven't changed
position with undesired HOTAS input and could be done in one left to right
scan.

Pilot

\begin{itemize}
  \item Radio - Set correctly
  \item Engines - both equally running and temps good
  \item Fuel - on target, balanced and check if externals dry
  \item Warning Panel - clear
  \item Lighting - as briefed
\end{itemize}

RIO

\begin{itemize}
  \item Radio - set correctly
  \item Warning Panel - clear
  \item INS \& DL functioning
  \item RWR - on and filter set as required
  \item ECM - set as required
  \item TGP - warming up if required
\end{itemize}

\subsubsection{Belly Check}

A belly check is an exterior check performed by a wingman, looking for damage
that the crew cannot see.

When damage is suspected should be also accompanied with specific moving part
checks, especially gear and tires. It's more usual on the return leg but can be
requested at any time.

The aircraft being checked should be thoroughly stable and level before
commencing. This is especially the case if any configuration (gear, flaps,
slats) changes are required for inspection as it then becomes quite dangerous.

The inspecting aircraft should report external stores, configuration of moving
parts and any signs of damage.

\newpage

\subsubsection{FENCE check}

A FENCE check is a directive to change the configuration of aircraft in a
flight into and out of a combat ready state. FENCE checks should be made at a
prebriefed point when entering proximity to enemy and leaving the same.

Fence checks have had various acronyms but cover the same things. Ensure your
flight isn't too fast coming back with their report. Only provide response
"Fenced in, 10.4" when all the checks are done.

\begin{enumerate}[
  align=parleft,
  itemsep=-0.25em,
  labelsep=0.25cm,
  leftmargin=*
]

  \item[\textbf{F}] Fuel Feeding, state, test all tanks
  \item[\textbf{E}] Emitters, use the acronym \textbf{TRAIL}

  \begin{enumerate}[
    align=parleft,
    itemsep=-0.25em,
    labelsep=0.25cm,
    leftmargin=*
  ]
    \setlength\parskip{0.3em}

    \item[\textbf{T}] TACAN. Check operation. A/A set as briefed
    \item[\textbf{R}] Radar. Mate radars at contract or prebriefed scan volumes
    \item[\textbf{A}] "ALE/ALR". Set ECM and RWR as required
    \item[\textbf{I}] IFF. Set modes, codes, as required
    \item[\textbf{L}] Lights. All exterior off
  \end{enumerate}

  \item[\textbf{N}] Navigation - INS, accumulated error, check for update.
    TACAN looks good, set as required

  \item[\textbf{C}] Chaff and Flares, set program, and Pilot operation,
    including live discharge check

  \item[\textbf{E}] Employment; Bomb release order, step, interval and stations
    and arming, Sidewinder cooling, Missile coolant begun, Gun Rate, TGP warmed
    and functional, A-A/A-G set. Secure Radio's if available

\end{enumerate}

\subsection{Scans}

Scanning is a required discipline that takes a lot of effort. Three types of
scan are recommended: \textbf{Positional} in relation to the flight,
\textbf{Instrument} scan for the aircraft and \textbf{Combat} scanning.

\subsubsection{Positional scan}

A pilot is required to be in their briefed formation at all times, in visual
range for the assumptions of visual to be workable.

\begin{itemize}

  \item Rate of climb and descent trimmed state (stability)

  \item Rate of closure and separation

  \item Relational positioning to element lead (briefed formation)

  \item Visibility to and from their element lead (skyline, marking, lights,
    too high/low, deep six, under wing etc)

\end{itemize}

\subsubsection{Instrument scan}

Pilots will follow an instrument scan every 10 seconds

\begin{itemize}
  \item Airspeed Indicator
  \item Attitude Indicator HSI
  \item Altimeter
  \item Vertical Speed Indicator
  \item Heading Indicator
  \item Ball/Slip indicator
\end{itemize}

The instrument scan prevents wandering from desired altitude, wandering off
course, allowing poor flight regimes to develop.

\subsubsection{Pilot Combat scan}

Pilots will conduct combat scans

\begin{itemize}
  \item Personal 360 scan
  \item Visual checks through their element lead in his blind spots
  \item Sensors RWR
  \item TID repeater if available
\end{itemize}

\subsubsection{RIO scan}

The RIO will conduct the following primary scans

\begin{itemize}
  \item Navigational checks
  \item TOT and fuel checks
  \item Radar scanning as required
  \item RWR threat check
\end{itemize}

The RIO will back up the pilot for these scans but not be the primary person in
charge or delegated to unduly.

\begin{itemize}
  \item Backup visual scan when able
  \item Backup positional reminders to pilot
  \item Backup Altitude block check
  \item Backup speed check/ rate of climb/descent check
\end{itemize}

Note: Good RIO's attempt 50\% outside cockpit scanning 50\% inside. This comes
with experience.
