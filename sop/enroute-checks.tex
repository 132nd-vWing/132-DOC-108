\section{EN ROUTE CHECKS \& SCANS}

\subsection{Checks}

\subsubsection{Ops Check}

An Ops check is a quick check over the aircraft that puts several reminders in
the front of the mind, notably fuel and switchology. It is most useful for
verifying quickly that switches you don't often visually check haven't changed
position with undesired HOTAS input. Ops check can be handled in a simple left
to right scan of the cockpit.

Pilot

\begin{itemize}
  \item Radio - Set correctly
  \item Engines - both equally running and temps good
  \item Altimeter - set correctly for current level (QNH/QNE)
  \item Wings - set and operating as required
  \item Fuel - on target, balanced and check if externals dry
  \item Warning Panel - clear
  \item Lighting - as briefed
  \item Air Source - BOTH
\end{itemize}

RIO

\begin{itemize}
  \item Radio - set correctly
  \item AWG-9 - controls set as required
  \begin{itemize}
    \item Stby, Bars, Tgt size, MLC, Aspect, Pulse range (for IFF range), Vc,
    P.gain, Erase, Mode, etc.
  \end{itemize}
  \item Altimeter - set correctly for current level (QNH/QNE)
  \item Warning Panel - clear
  \item INS \& DL functioning
  \item Fuel - on target, quantity used
  \item Navigation - check Mode, selected WP, and parameters (speed, distance,
  time)
  \item RWR - on and filter set as required
  \item ECM - set as required
  \item TGP - warming up if required
\end{itemize}

\subsubsection{Belly Check}

A belly check is an exterior check performed by another aircraft, looking for
damage that the crew is unable to see.

When damage is suspected, the belly check should be also accompanied with a
specific moving part such as gear and tires. It's more common on the return leg
but can be requested at any time.

The aircraft being checked should be in stable and level flight before
commencing, and pilots need to pay close attention for any restrictions
to avoid over speeding gear, flaps or other control surfaces.

The inspecting aircraft should report external stores, configuration of moving
parts and any signs of damage.

\newpage

\subsubsection{FENCE check}

A FENCE check is a directive to change the configuration of aircraft in a
flight into and out of a combat ready state. FENCE checks should be made at a
pre briefed point when entering proximity to enemy and leaving the same.

Fence checks have had various acronyms but cover the same things. Ensure your
flight isn't too fast coming back with their report. Only provide response
\textit{"Fenced in, 10.4"} when all the checks are done.

\begin{enumerate}[
  align=parleft,
  itemsep=-0.25em,
  labelsep=0.25cm,
  leftmargin=*
]

  \item[\textbf{F}] Fuel Feeding, state, test all tanks
  \item[\textbf{E}] Emitters, use the acronym \textbf{TRAIL:}

  \begin{enumerate}[
    align=parleft,
    itemsep=-0.25em,
    labelsep=0.25cm,
    leftmargin=*
  ]
    \setlength\parskip{0.3em}

    \item[\textbf{T}] TACAN, check operation
    \item[\textbf{R}] Radar, mate radars at contract or pre-briefed scan
      volumes
    \item[\textbf{A}] ALE/ALR, set ECM and RWR as required
    \item[\textbf{I}] IFF, set modes and codes as required
    \item[\textbf{L}] Lights, set Exterior Lights Master Switch - \textbf{OFF}
  \end{enumerate}

  \item[\textbf{N}] Navigation, INS, accumulated error, check for update.
    TACAN looks good, set as required

  \item[\textbf{C}] Chaff and Flares, set program, and Pilot operation,
    including live discharge check

  \item[\textbf{E}] Employment, Bomb release order, step, interval and stations
    and arming, Sidewinder cooling, Missile cooling, Gun Rate, TGP warmed
    and functional, A-A/A-G set. Secure Radios if available

\end{enumerate}

\subsection{Scans}

Scanning is a required discipline that takes a lot of effort. Three types of
scan are recommended: \textbf{Positional} scan in relation to the flight,
\textbf{Instrument} scan for the aircraft and \textbf{Combat} scanning.

\subsubsection{Positional scan}

A pilot is required to be in their briefed formation at all times and remain in
visual range of the flight.

\begin{itemize}

  \item Rate of climb and descent trimmed state (stability)

  \item Rate of closure and separation

  \item Relative positioning to element lead (briefed formation)

  \item Visibility to and from their element lead (skyline, conning, lights,
    too high/low, deep six, under wing etc)

\end{itemize}

\subsubsection{Instrument scan}

Pilots will perform an instrument scan every 10 seconds

\begin{itemize}
  \item Airspeed Indicator
  \item Attitude Indicator
  \item Altimeter
  \item Vertical Speed Indicator
  \item Heading on HSD or BDHI
  \item Ball/Slip indicator
\end{itemize}

The instrument scan prevents drifting from desired altitude, drifting off
course, or allowing poor flight regimes to develop.

\subsubsection{Pilot Combat scan}

Pilots will perform combat scans

\begin{itemize}
  \item Personal 360 scan
  \item Visual checks across their flight, especially of wingmen's blind spots
  \item Sensors, RWR
  \item TID repeater, if available
\end{itemize}

\subsubsection{RIO scan}

The RIO will conduct the following primary scans

\begin{itemize}
  \item Navigational checks
  \item TOT and fuel checks
  \item Radar scanning as required
  \item RWR threat check
\end{itemize}

The RIO shall assist the pilot with the following scans, but will not hold
primary responsibility for their completion:

\begin{itemize}
  \item Backup visual scan when able
  \item Backup positional reminders to pilot
  \item Backup Altitude block check
  \item Backup speed check, rate of climb, or descent check
\end{itemize}

RIOs should develop and maintain a 50/50 split of inside/outside scanning.

