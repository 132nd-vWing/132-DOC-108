\section{CREW ROLES \& RESPONSIBILITIES}

With the advent of a crewed combat aircraft to DCS that have specific
'divisions of labor', there are some explanations required on standardizing
roles. This SOP will address answers on who does what. For example, both crew
members have a weapon fire button that does the same thing. Who gets to push it
first?

An additional question is how the 1970-2000 US Navy practices fit into 132\nd
doctrine, whether post Tomcat lifespan processes will be included or improved
on, whether NATO, USAF or USN (or other Nationalities) processes have a place,
and how to convert modern processes to 'end of life' weapons and technology.

The basic guiding principal is that these all become 132\nd vFW processes, with
the primary guidance of "compatibility with existing 132\nd Squadrons first and
foremost". Next comes the plane documentation with any deviations from game and
real life outlined. Then with process we standardize Natops over USAF and last
USN. Whilst USN documents may be used, they have no seniority implied and
require to be examined for interoperability between squadrons in order to
harmonize and simplify multiple squadrons working together. Typically, this is
seen as Navy terms and processes being removed, where appropriate, and fitting
the USAF terms, which despite being an ugly looking prospect, will provide less
confusion in the long run. It must be remembered, that the USN and USAF did not
always work well together pre 2000 and there is no reason to repeat their
mistakes!

Not all the original Tomcat processes will survive. Unfortunately, this is not a
historically accurate simulator, as appealing as it would be to try and model it
so. Therefore if a scenario in 1986 included a certain technique, we will
continue to use modern approaches within the 132\nd and this may require some
adaptation to SOPs which can always be overridden with specific scenario SPINS.

\subsection*{Mission Commander}

A \textbf{Mission Commander (MC)} is a crew role in the Flight Lead's aircraft.
An MC can be either a Pilot or RIO. It is based on the US Navy term of the same
name.  This is a role taken on by one of the Flight Lead's crew. The MC will
plan and brief and execute the Flight's Mission, as provided in the Air Tasking
Order (ATO). This role is required in order to delineate the person with
overall responsibility for flight planning and Mission success, whilst being in
the same aircraft with another person using the same callsign. It usually makes
sense for the RIO to be MC, as they are more tactically tuned to the mission
with higher SA, but they will not always be available, so it should be
specifically allocated.

\subsection*{Pilot}

The \textbf{Pilot} is the captain of the plane and oversees the plane and its
crew, with special emphasis on safety, and is especially in charge of the
aircraft's decision-making process. Whilst Pilots naturally have the control of
the plane as their sole responsibility, they do answer to the higher authorities
of the Mission Commander and Squadron hierarchy. In this SOP the Pilot takes
'Advisories' and 'Directives' from the RIO, Mission Commander and Weapons
Director in the same way that a Wingman would call for a break to defeat a
missile. This is consistent with any historical or modern day behaviour. In
actual practice, the Pilot will respond to the RIO's advisories as directives as
though from a superior authority, regardless of Rank. Any 'friction' in decision
making is only a human artefact. If the Pilot refuses to take the RIO's advice,
refrain from discussion until debrief along with any TacView, and supplementary
information you may have access to. A damaged Pilot-RIO relationship is of great
concern.

\subsection*{RIO}

The \textbf{Radar Intercept Officer} is the back-seat crew of the F-14B Tomcat.
He has a specific role to conduct the operation of the AWG-9 Radar, DataLink,
Countermeasures, Tactical decision making, TV equipment, LITENING Pod, TARPS,
Stores and jettison mechanics.

As the title suggests, 'Radar Intercept Officer', has a core role during an
intercept and BVR air to air combat. The extent of their authority during this
phase is that they will be the person making Tactical decisions up until they
hand over control verbally to the Pilot.

This exchange of handover is quite critical to get right as it could lead to
both crews being heads down in a visual fight. The RIO's 'leadership term' ends
with "Tally" from the Pilot.

At any time, if the RIO feels he no longer has useful SA, he can hand the radar
to the Pilot with "Your Radar". The Pilot will respond "My Radar" and hand this
back at the termination of his control.

The RIO exists as an extension to the pilot. In testing, nearly every time a
Pilot benefits more from a average RIO than the AI counterpart. Additionally,
the Jester AI is severely limited.

\subsection*{Flight Planning}

This SOP will state a simple directive that all lead aircraft must be filled
with a Human RIO first. In this configuration, the lead aircraft carries some
authority with the extra crew, some extra "head space", a Tactical advisor and
of course all the benefits that Jester AI cannot give us, such as LANTIRN pod
use, FAC(A), clever use of Manual TWS and radar use and spare ears and mouth
for Tactical calls with the AWACS controller.

Allocations of Pilot to Rio and plane to plane will depend on the Mission Host.
Whilst "power couples" are more effective, it also builds restrictions.

\subsection*{Flight Lead}

Lead is a term usually applying to a single person in an aircraft. However, in
a crewed aircraft any of the crew could respond to a message. It is important
therefore to continue to use effective communication practices and be clear
with the usage of the own ships callsign or side number so that multiple voices
are not confusing.

Furthermore, we delineate Pilot and RIO as "Flight and Tactical". There are no
limitations to Pilot-RIO superseding contracts if both parties are in agreement
before a flight, as long as they do not supersede a Squadron SOP regarding
safety, mission effectiveness, and the like.

Further detail can be found in \fullref{sec:communications-radio-naming}
