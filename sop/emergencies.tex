\section{EMERGENCIES}

\subsection{Radio Failure}

A pilot whose radios fail while in close formation will manoeuvre within safe
close parameters and then Push forward in front of lead whilst switching on and
off lighting randomly to indicate Systems malfunction and radio failure. They
will fall back into formation if continuing or if he is RTB by lowering the
Hook and breaking off sharply in front of lead.

Once the element lead has been established that element will navigate back to
home plate or the nearest divert airfield and handle all radio communications
necessary to bring the other pilot back to final approach, at which point the
lead will perform a go-around to allow the damaged plane to land first.

For CV Ops Radio failures, the established external sign is you should pass the
port side of the carrier at low level with the hook down, shaking your wings
and blinking lights

\subsection{Flat spin (F-14B)}

The SOP for 108\th for upright Flat spins are as follows:

\begin{itemize}

  \item Ensure you are actually in a flat spin. IAS should be <100kts, the yaw
    indicator should be pegged to one side, the direction of the spin, and you
    should be rapidly spinning and falling with a slight nose bob. Your stick
    and rudder should be neutralised.

  \begin{itemize}

    \item If for any reason the spin hasn't fully materialised, take your hands
      and feet off all controls immediately and count to 5.

    \item Reassess your departure then

  \end{itemize}

  \item Once verified, check altitude. If you are beneath 10,000ft AGL,
    immediately prepare to eject by

  \begin{itemize}
    \item Releasing the canopy emergency jettison handle
    \item Commanding eject once the canopy is clear
  \end{itemize}

  \item If you are over 10,000 feet and react fast enough, immediately follow
    the NATOPS which is

  \begin{itemize}

    \item Idle both throttles

    \item Put the Rudder into maximum \textbf{opposite} direction of the spin
      (don't look at the ball, just the slip indicator which is hiding)

    \item With the stick centred, push the stick all the way \textbf{forwards
      and hold}

    \item Then move the stick \textbf{into} the direction of the spin and hold
      it at maximum deflection forward and into the spin.

  \end{itemize}

  \item You are now crossed control. This is how to get INTO a flat spin
    normally, hence step 1 is to ensure you are actually really in one. You
    have to now start praying, because what happens next is sometimes down to
    luck.

  \item The Aircraft will slow in the spin and the nose will oscillate down and
    up more. At some point, the nose will end up pointed mostly down and the
    IAS will jump to life. If the upwards oscillation has lost enough energy
    and you think you are on your last revolution, RELEASE ALL CONTROLS!

  \item At >100kts gently stabilise the yaw and any rotation and pull smoothly
    to the horizon with a maximum of 17 units of AoA.

  \item If in full recovery mode by 5,000 feet and no progress has been made,
    initiated the ejection. Inverted Flat spins have not been tested.

\end{itemize}

Inverted Flat spins have not been tested.

\newpage

\subsection{In flight engine restart (F-14B)}

The SOP currently for an in-flight engine restart is

\begin{itemize}

  \item Idle cutoff the engine

  \item Assess the damage and possible causes whilst establishing control of
    the airplane in coordinated flight

  \item If no Fire is suspected, and airspeed is good, follow the Cross-Bleed
    Airstart checklist

  \begin{itemize}
    \item \href
      {http://www.heatblur.se/F-14Manual/emergency.html\#engine-airstart}
      {http://www.heatblur.se/F-14Manual/emergency.html\#engine-airstart}
  \end{itemize}

  \item Land as soon as is practical

\end{itemize}

