
\subsection{References}

\textbf{\href
  {https://www.cnatra.navy.mil/local/docs/pat-pubs/P-825.pdf}
  {P-825 CNATRA - All Weather Intercept}}

\narrow{
  \begin{enumerate}[
    align=parleft,
    itemsep=-0.25em,
    labelsep=1.25cm,
    leftmargin=*
  ]

    \item[\textbullet~Ch2] Air Intercept Control

    \item[\textbullet~Ch3] Comms (skim)

    \item[\textbullet~Ch7] Target Aspect Control (stern Conversion, more
      important for civilian)

    \item[\textbullet~Ch12] MRM + SRM employment (skim, ignore ranges)

    \item[\textbullet~Ch13] Advanced Intercepts (good understanding of
      terminology but ignore ranges)

  \end{enumerate}
}

\textbf{\href
  {https://cloud.132virtualwing.org/s/DdLFQX4LLTExij2}
  {132-TTP-Air-to-Air-1 v1.0}}

Note: This SOP supersedes A-A TTP based on the different and more specific
Tomcat detail.

\subsection{Terminology}

\renewcommand\arraystretch{1.5}\begin{tabularx}{\textwidth}{lX}

  \textbf{AIC} & Air Intercept Controller a term for any GCI or AWACS be it
    ship, airborne or ground based
  \\
  \textbf{CT} & Continuation Training. Training after beyond MQT checkout.
  \\\textbf{MC} & Mission Commander. The crew member in the lead aircraft with
    ultimate responsibility of the execution of the mission
  \\\textbf{Mission} & The objectives that are set and the resultant briefing
  given by the Mission Commander
  \\
  \textbf{Heavy Group} & A bogey group within 3nm that have three or more
  aircraft
  \\
  \textbf{NLT} & Acronym for "No Later Than", meaning do this beforehand or at
  the very latest
  \\
  \textbf{MAR} & Acronym for Minimum Abort Range - Defined fully elsewhere, the
  range at which the aircraft can turn around and exit the fight without
  entering the lethal weapons range of the bandit (his Range Max 2 or No Escape
  Zone (NEZ))
  \\
  \textbf{Pri} & Abbreviation for the Fore Radio tuned to the flight channel
  \\
  \textbf{Aux} & Abbreviation for the AFT radio tuned to the flight-external
    controlling agency, e.g. "AWACS"
  \\
  \textbf{FL} & Abbreviation for Flight Lead
  \\
  \textbf{WM} & Abbreviation for Wingman
  \\
  \textbf{WM RIO (Aux)} & Comms examples for a Wingman RIO speaking on the Aft
  radio tuned to Agency
  \\
  \textbf{BRAA} & Acronym for Bearing, Range, Altitude, Aspect and is and
    relative to own-ship position. Aspect is only provided by the AIC. Flights
    give BRAA and omit aspect.
  \\
  \textbf{Bullseye (BE)} & A common reference point on the map, from which
    locations can be referenced and understood by all regardless of own-ship
    location. \textit{"Contact, Bullesye, 360, 10, 12,000"} identifies a
    contact 10NM north of the bullseye location at 12,000 ft
\end{tabularx}


\newpage

\boxed{
  Commit criteria in every mission \textbf{must} be clearly known

  We will illustrate shoot and crank \textbf{NLT 35nm} but range may be
  increased by Mission criteria

  In fully client-crewed aircraft, the RIO has the Tactical Control + comms

  Persistently HOT Bandits at 40nm (MELD) should be Identified by the flight
  as HOSTILE for game purposes and to prevent undue delay to the
  Identification.

  RIO will conduct all these calls unless the aircraft is manned only with a
  client pilot and Jester AI

  RIO will provide steering instructions to the Pilot for an intercept and
  manage the Timeline and Employment
}
