
\subsection{References}

\textbf{\href
  {https://www.cnatra.navy.mil/local/docs/pat-pubs/P-825.pdf}
  {P-825 CNATRA - All Weather Intercept}}

\narrow{
  \begin{enumerate}[
    align=parleft,
    itemsep=-0.25em,
    labelsep=1.25cm,
    leftmargin=*
  ]

    \item[\textbullet~Ch2] Air Intercept Control

    \item[\textbullet~Ch3] Comms (skim)

    \item[\textbullet~Ch7] Target Aspect Control (stern Conversion, more
      important for civilian)

    \item[\textbullet~Ch12] MRM + SRM employment (skim, ignore ranges)

    \item[\textbullet~Ch13] Advanced Intercepts (good understanding of
      terminology but ignore ranges)

  \end{enumerate}
}

\textbf{\href
  {https://cloud.132virtualwing.org/s/DdLFQX4LLTExij2}
  {132-TTP-Air-to-Air-1 v1.0}}

Note: This SOP supersedes A-A TTP based on the different and more specific
Tomcat detail.

\subsection{Terminology}

\renewcommand\arraystretch{1.5}\begin{tabularx}{\textwidth}{lX}

  \textbf{AIC} & Air Intercept Controller a term for any GCI or AWACS be it
  ship, airborne or ground based
  \\
  \textbf{CT} & Continuation Training. Training after beyond MQT checkout.
  \\
  \textbf{MC} & Mission Commander. The crew in the Lead's flight with ultimate
  responsibility of the execution of the mission
  \\
  \textbf{Mission} & The objectives that are set and the resultant briefing
  given by the Mission Commander
  \\
  \textbf{Heavy Group} & A bogey group within 3nm that have three or more
  aircraft
  \\
  \textbf{NLT} & Acronym for "No Later Than", meaning do this beforehand or at
  the very latest
  \\
  \textbf{MAR} & Acronym for Minimum Abort Range - Defined fully elsewhere, the
  range at which the aircraft can turn around and exit the fight without
  entering the lethal weapons range of the bandit (his Range Max 2 or No Escape
  Zone (NEZ))
  \\
  \textbf{Pri} & Abbreviation for the Fore Radio tuned to the flight channel
  \\
  \textbf{Aux} & Abbreviation for the AFT radio tuned to agency control i.e.
  'Magic'
  \\
  \textbf{FL} & Abbreviation for Flight Lead
  \\
  \textbf{WM} & Abbreviation for WingMan
  \\
  \textbf{WM RIO (Aux)} & Comms examples for a Wing Man RIO speaking on the Aft
  radio tuned to Agency
  \\
  \textbf{BRAA} & Acronym for Bearing, Range, Altitude, Aspect. Aspect is only
  provided by the AIC. Flights give BRAA and omit aspect
  \\
  \textbf{Bullseye (BE)} & A common reference point on the map, so locations
  can be discussed from an absolute reference rather than a flight reference
  (and thus used by all). Bullseye 0 for 0 at ten
\end{tabularx}


\newpage

\boxed{

  \textbf{All calls in BRAA} despite Broadcast control (until further notice)

  Mission criteria \textbf{must} clearly identify engagement ranges for Phoenix
  due to variance

  Commit criteria in every mission \textbf{must} be clearly known

  We will illustrate shoot and crank \textbf{NLT 35nm} but range may be
  increased by Mission criteria

  Differences in Crewed aircraft responsibilities (see SOP) RIO has the
  Tactical Control + comms

  Persistently HOT Bandits at 40nm (MELD) should be Identified by the flight
  as HOSTILE for game purposes and to prevent undue delay to the
  Identification.

  RIO will conduct all these calls unless manned as "single ship" (therefore
  Pilot needs to know)

  RIO will provide steering cues to the Pilot for an intercept and manage the
  Timeline and Employment
}
