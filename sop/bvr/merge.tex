\subsection{Merge follow on}

\subsubsection*{Simple Definitions}

\begin{itemize}

  \item AREO: Azimuth, Range, Elevation, Overtake (closure) calls
  \item Merge: Within 3nm
  \item Tally: Proword for "I see the Bandit"
  \item Visual: Proword for see my Wingman/friendly
  \item Nojoy: Proword for can't Tally
  \item Blind: Proword for can't see WM or friend
  \item 4 Assumptions: Clean, Naked, Nojoy and Visual

\end{itemize}

\subsubsection*{General}

This particular phase requires careful coordination and precise Comms snippets.

\subsubsection*{Process}

From ten miles the RIO should begin to provide SA calls until the Pilot gets
Tally. These are called AREO calls. Once the Pilot has Tally, the RIO should
sanitise, focus on Wingman, or use Speed callouts and anything useful to backup
the pilot.

If the RIO loses Radar SA close to the merge, he should prompt the Pilot to
take the radar. The RIO should prompt to return the radar once the immediate
area has been sanitised and the engagement is complete.

The Flight should callout on the 4 assumptions whenever they change during the
intercept and ACM. Thus if Spiked, call it out, if radar contact, call it out
if Tally, call it out and if Blind, call it out on the Flight channel.

Who calls Blind and Tally?

The Pilot. However, he should repeat this on ICS first to allow the RIO to
focus his gaze and help him first.

\subsubsection*{Example}


\textbf{FL RIO (ICS):} "On the nose, 4miles, 3 high, 585 over"

\textbf{FL RIO (ICS):} "You have the radar"\\
\textbf{FL Pilot (ICS):} "I have the radar" (selects PAL)

\textbf{FL Pilot (Pri):} "Blind, Tally 1, 3 o'clock High"\\
\textbf{FL RIO (Pri):} "2 is at your 9 o'clock high" \\
or \\
\textbf{WM Pilot (Pri):} "2, Your 9 o'clock high"
