\subsection{Merge follow on}

\subsubsection*{Simple Definitions}

\begin{itemize}

  \item AREO: Azimuth, Range, Elevation, Overtake (closure) calls
  \item Merge: Within 3nm
  \item Tally: Proword for "I see the Bandit"
  \item Visual: Proword for see my Wingman/friendly
  \item Nojoy: Proword for "I don't see the Bandit"
  \item Blind: Proword for "I can't see WM / Friendly"
  \item Clean: No sensor (here: own radar) information on a GROUP of interest
  \item Naked: No (relevant) Radar Warning Receiver indications
\end{itemize}

\subsubsection*{General}

This particular phase requires careful coordination and precise and efficient
comms in Brevity Code.

\subsubsection*{Process}

Within ten miles the RIO should begin to provide SA calls until the Pilot gets
Tally. These are called AREO calls. Once the Pilot has Tally, the RIO should
sanitise, focus on Wingman, or use Speed callouts and anything useful to backup
the pilot.

If the RIO loses Radar SA close to the merge, he should prompt the Pilot to
take the radar with "Your Radar", Once the immediate area has been sanitised
and the engagement is complete the RIO can take back the radar with "My Radar"
as outlined in \fullref{sec:radar-and-combat-contract}

When fighting Within Visual Range (WVR), the flight members should update each
other about any status changes of the above 6 Definitions for both positive and
negative visual and/or sensor contacts.

Any acquisition or loss of eye, radar or RWR contact with friends and foes
alike should be called on the Flight channel to build and maintain common SA of
the developing 'Furball' (= 'Dogfight').

To avoid Flight frequency congestion (as far as possible), Pilot/RIO crews
should first confer internally over ICS if the crew partner sees a contact that
the other partner has lost.

Aircraft external information exchange on the Flight channel should primarily
be by the pilots - but could also be a RIO warning a WM about Bandit on WM's
six.

It is advisable to discuss with your Pilot/Rio prior to your flight regarding
the use of ICS "Hot Mic" should you encounter WVR engagements to enable the
free flow of information during this critical flight phase without starting a
discussion at an inappropriate time.

\subsubsection*{Example}


\textbf{FL RIO (ICS):} "On the nose, 4 miles, high, 585 over"

\textbf{FL RIO (ICS):} "Your Radar"\\
\textbf{FL Pilot (ICS):} "My Radar" (selects PAL)

\textbf{FL Pilot (Pri):} "Blind, Tally 1, 3 o'clock high"\\
\textbf{FL RIO (Pri):} "2 is at your 9 o'clock high" \\
or \\
\textbf{WM Pilot (Pri):} "2, your 9 o'clock high"
