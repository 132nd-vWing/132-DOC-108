\section{COMMUNICATION}

\subsection{General}

132\nd provides an additional guidance on Brevity and communication in the form
of \href{https://cloud.132virtualwing.org/s/4TPtFDbmPbw9oWM}{TTP-4 Brevity}.
For 108\th purposes we have some extra items to go over regarding the roles of
the two crew.

\subsubsection{ICS}

Hot mic vs Cold mic: This choice remains at the discretion of individual flight
crews. All members should be prepared for "Push to Talk" and have bindings
setup accordingly. Should either crew member request a change to Push to
Talk ICS, you must grant this request.

Operational 'chit-chat' is noticeably high on ICS due to the less formal
relationship; however it is \textbf{strongly advised to behave in a formal
manner} to reduce PTT open time and maximize radio listening time. Overtalking
external radios with ICS comms is operationally "dangerous" (in simulation
terms), requires repeat transmissions and causes stress and miscommunication.

\subsubsection{RADIO naming conventions}
\label{sec:communications-radio-naming}

The Pilot's radio is called the \textbf{Forward} radio and the RIO's Radio is
called the \textbf{Aft} Radio.

Communications with 'agencies' should be performed on the Aft radio under
normal circumstances due to it's wider frequency range (VHF and UHF) and it's
locality to the main radio operator of the aircraft, the RIO. Additionally, the
Aft radio is the only radio in the aircraft configured for ADF.

By default, Inter-Flight communications are made from the Forward radio. Note:
Jester, when tuning, can set FM and not AM, double check that!

\subsection{Response priority}

Pilots \textit{generally} respond to calls from other pilots. RIOs
\textit{generally} respond to calls from other RIOs. The easiest way to discern
context is from which radio the message is on, given the Pilot mostly uses the
Flight comms on Forward radio and RIO on agencies. Where confusion may arise, it
has become standard for intra-flight comms to use "Alpha" for the pilot and
"Bravo" for the RIO, combined with the aircraft's number:

\narrow{
  \textbf{Lead RIO (Fwd):} Two Bravo, One Bravo, report ready to copy revised
  NAVGRID settings! \\
  \textbf{Wing RIO (Fwd):} Two Bravo, ready (to copy)
}

General Rules:

\begin{itemize}

 \item Pilots have responsibility for movement, formation, visual arena, and
 ACM.

 \item RIOs have responsibility for the Tactical and BVR Intercept and
 Navigation. During Intercepts, RIO will also have the flight comms for
 tactical.

 \item RIOs have default responsibility of ATC unless pre-arranged
 otherwise between the crew.

\end{itemize}

Flight communications must be verified between the flight lead and all
flight members on the pre-briefed Forward channel and Aft agency channel prior
to requesting Taxi clearance.

There are two types of legitimate messages on any radio:

\begin{itemize}
 \item Directive
 \item Informative
\end{itemize}

Directive calls are orders and \textbf{require} an action and must be responded
to. It is usually enough to respond with your abbreviated callsign, with a smart
acknowledgement;"Two". For complex directives or critical details, pilots are in
the habit of "Reading back" the details to ensure they heard it correctly.

Informative calls are calls which help build knowledge and SA and do not require
acknowledgement. Such calls can be critically important alerts, or fairly
low-key information.

Directives \textbf{must} always be acknowledged succinctly, in flight order.
This allows the flight lead to notice a non-responsive member.

\subsection{Frequency Change Procedures and Radio Check}

\begin{itemize}

  \item A direction to '\textbf{push}' to a certain frequency
  (or preset) need not be acknowledged.

 \item A direction to '\textbf{go}' to a certain frequency (or
  preset) must be acknowledged by all flight members.

\end{itemize}

After the Flight has tuned to a new frequency, a radio check using that radio
is advised. This will be initiated by the flight lead as follows:

\narrow{
    \textbf{Lead (Tuning Radio)}: Callsign, check (Aft|Forward) \\
    \textbf{Wing (Tuning Radio)}: Two
}

\subsection{The Pilot-RIO relationship and communications}

The Pilot-RIO relationship has an awkward situation where the plane's captain
takes directives from a junior agency. This needs to be specifically addressed.
During the Radar Intercept the Pilot is not completely aware of the detail of
the picture that the RIO sees. The RIO has the role of conducting intercepts up
to the merge The RIO will direct the pilot for the intercept using some basic
prowords

\begin{itemize}
  \item Come Left/Right X degrees (naval terminology would prefer port and
  starboard respectively)

  \item Nose Up/Nose Down X degrees

  \item Descend/Climb Angels X

  \item Steady Up! (roll wings level and anticipate missile release)

  \item Offset (Specific Intercept geometry in relation to an aspect, but
  ultimately a heading to steer)

  \item Centre the T - Cue the nose to the missile T (anticipate the shot)

  \item Your radar, (acknowledged with "My Radar") - hand off the radar
  to the other crew member, this is useful in BFM where the Pilot may benefit
  from the various front-seat focused radar modes and the RIO is better served
  monitoring instruments and/or maintaining visual contact.

  \item My Radar, (acknowledged with "Your Radar") - request radar control from
  the other crew member.

\end{itemize}

In most other cases, communications will attempt to model a Pilot-Wingman
relationship when using prowords and terminology.

The Pilot will attempt to follow the RIO's guidance as best as possible, this
is their relationship. Vetoing RIO decisions is a possibility, if flight safety
is in question.

This SOP does not supersede established working Pilot-RIO relationships but
provides a guide for unfamiliar pairs.

\subsection{SA calls and SEM}

See \fullref{sec:BVR TIMELINE} and \fullref{sec:ACM} for how situational
awareness calls are made on the Intercom.

