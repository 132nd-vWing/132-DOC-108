\section{COMMUNICATION}

\subsection{General}

132\nd provides an additional guidance on Brevity and communication in the form
of TTP-4 Brevity For 108\th purposes we have some extra items to go over
regarding the roles of the two crew.

\subsubsection{ICS}

Hot mic vs Cold mic. Settings will be individual based on the quality of each
crews mic and surrounding room \textit{noise pollution}. Verify the setting and
volume are good.

Operational 'chit-chat' is noticeably high on ICS due to the less formal
relationship; however it is \textbf{strongly advised to behave in a formal
manner} to reduce PTT open time and maximize radio listening time. Overtalking
external radios with ICS comms is operationally "dangerous" (in simulation
terms), requires repeat transmissions and causes stress and miscommunication.

\subsubsection{RADIO naming conventions}

The Pilot's radio is called the \textbf{Forward} radio and the RIO's Radio is
called the \textbf{Aft} Radio.

Communications with 'agencies' should be performed on the Aft radio under
normal circumstances. This is due to the lower frequency range it has capable
and sitting near the Tactical agency controller, the RIO. This radio also has
the only workable ADF control, so it is close to the Navigator. Flight
Communications are by default made from the Forward radio. Note: Jester, when
tuning, can set FM and not AM, double check that!

\subsection{Response priority}

Pilots generally respond to calls from other Pilots, RIO's will generally
respond to calls from other RIO's in the context of what stage of flight the
aircraft is in or the directive being applied. The easiest way to discern
context is by what radio the message is on, given the Pilot mostly uses the
Flight comms on Forward radio and RIO on agencies. Where confusion could arise,
pilot "callsigns/nicknames" have been used operationally. You may also use "2's
RIO" or "Lead's Pilot" which is a much simpler and clearer and operationally
secure method. Much of this will develop as relationships and cannot be managed
tightly. General rules:

\begin{itemize}

 \item Pilots have responsibility for movement, formation, visual arena
 communication, flight communication and ACM.

 \item RIO's have responsibility for the Tactical and BVR Intercept and
 Navigation. During Intercepts, RIO will also have the flight comms for
 tactical.

 \item RIO shall take default responsibility of ATC unless pre-arranged
 otherwise between the crew.

\end{itemize}

Intra-flight communications must be established between the flight lead and all
flight members on the pre-briefed channel or frequency before taxi.

There are two types of legitimate messages on any radio:

\begin{itemize}
 \item Directive
 \item Informative
\end{itemize}

Directive calls are orders and \textbf{require} an action and must be responded
to. It is usually enough to respond with your abbreviated callsign, with a smart
acknowledgement;"Two". For complex directives or critical detail Pilots are in
the habit of "Reading back" the detail to ensure they heard it correctly.

Informative calls are Calls that help build knowledge and SA and do not require
acknowledgement. Such calls can be critically important alerts, or fairly
low-key information. Pilots \textbf{must} always respond to Flight Lead in
numerical flight order. This allows the flight lead to notice a non-responsive
member.

Directives \textbf{must} always be acknowledged smartly, and in order!

\subsection{Frequency Change Procedures and Radio Check}

\begin{itemize}

 \item A direction from Flight Lead to 'push' to a certain frequency (or
 preset) should not be acknowledged.

 \item A direction from Flight Lead to 'go' to a certain frequency (or preset)
 must be acknowledged by all flight members.

\end{itemize}

After the Flight tuned to a new frequency, a radio check is advised. This will
be initiated by the flight lead as follows:

\textbf{FL}: 'Callsign, check Aft (or forward)' - say this on the new
frequency\\
\textbf{WM}: '2' - say this on the new frequency, \textbf{do not reply with
anything other than your number!}

\subsection{The Pilot-RIO relationship and communications}

The Pilot-RIO relationship has an awkward situation where the plane's captain
takes directives from a junior agency. This needs to be specifically addressed.
During the Radar Intercept the Pilot is not completely aware of the detail of
the picture that the RIO sees. The RIO has the role of conducting intercepts up
to the merge The RIO will direct the pilot for the intercept using some basic
prowords

\begin{itemize}
 \item Come right/Left X degrees (naval terminology would prefer port and
 starboard)

 \item Nose down/Nose Up X degrees

 \item Descend/Climb Angels X

 \item Steady Up! (roll wings level and anticipate missile release)

 \item Offset (Specific Intercept geometry in relation to an aspect, but
 ultimately a heading to steer)

 \item Centre the T - Cue the nose to the missile T (anticipate the shot)

 \item Pilot's radar - Cue that RIO is heads up and radar is less valuable
 than eyes

 \item I have the radar/give me the radar - RIO believes that 'heads down' is
 a better tactical option

\end{itemize}

In most other cases, communications will attempt to model a Pilot-Wingman
relationship when using prowords and terminology.

The Pilot will attempt to follow the RIO's guidance as best as possible, this
is their relationship. Vetoing RIO decisions is a possibility, if safety is in
question.

This SOP does not supersede established working Pilot-RIO relationships but
provides a guide for unfamiliar pairs.

\subsection{SA calls and SEM}

See \fullref{sec:BVR TIMELINE} and \fullref{sec:ACM} for how situational
awareness calls are made on the Intercom.

