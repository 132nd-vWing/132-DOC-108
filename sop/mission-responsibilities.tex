\section{MISSION PLANNING PROCESS}

\subsection{Responsibilities}

The responsibility for mission planning a flight rests with the Mission
Commander. These duties may be delegated to other flight members though checking
and verification is shared by all flight members. The Mission Commander may be
the RIO or Pilot in the Lead Flight aircraft. The Mission Commander does not
interfere with normal crew responsibilities during flight or supersede plane
safety but fits into the combined output of the single crew working in harmony.

"Lead" is the Mission Commanders aircraft. The responsible crew member in that
plane for leadership decisions will depend on the type of activity, with the
Pilot making calls regarding Safety, Positioning and Visual area calls and the
RIO taking tactical, radar and heads down based decisions. The crew together
will harmonize through practice and training to become one single entity to
become the 'Lead aircraft'. Wherever "Lead" is mentioned, this refers to the
combined crew output.

It is the responsibility of all pilots and RIOs in the Flight to verify mission
parameters to ensure no calculation mistakes or errors in judgement have been
made in the mission plan. Items to consider include but are not limited to fuel
requirements, Coordinate accuracy, take-off and landing data, map-and-route
preparation, and communications regimes.

\subsection{Mission Prerequisites and Objectives}

Mission objectives should be well defined and understood by all members of the
flight. Each flight might contain multiple objectives covering training,
currency, and mission specific needs.

Mission prerequisites include skills and capabilities required in order to
fulfil the mission. The mission cannot be undertaken unless all these
prerequisites are met, and should one change status to unmet for any reason
the mission must be aborted, and this information communicated to the mission
commander at the earliest possible opportunity to re-planning and re-tasking.

Training objectives include detailed expected outcomes about what is being
trained. This should be clear and precise. Vagueness will only lead to
ineffective training outcomes and wasted time. Currency objectives aim to
ensure each pilot remains current on certain skills. Some skills to keep
current include:

\begin{subsectionenumerate}
  \item Low visibility recovery
  \item Night recovery
  \item Manual recoveries
  \item Carrier recoveries
  \item Unfamiliar Airfields procedures
  \item Airborne refueling
  \item Lesser Navigation methods (ADF, TCN offset, INS offset)
  \item ACM/SEM
  \item Intercept/BVR
  \item CAS
  \item FAC(A)
  \item Targeting Pod use
  \item Unguided Weapons
\end{subsectionenumerate}

\subsection{Material}

The following should be prepared and reviewed before any flight:

Mission data card (MDC). Where possible make the data card DCS Kneeboard
compatible to accommodate pilots using Virtual Reality headsets. The contents
make up a minimal subset of the e-brief and flight bag. At a minimum this
should include:

\begin{subsectionenumerate}
  \item 'Go, No-Go' criteria. Intel updates, supporting mission cancellations,
    pilot unavailability.

  \item Flight Contract

  \item Relevant weight/stores/weather related speeds

  \item Map and target pictures

  \item F-14B Tomcat checklist. The Mission Commander should ensure that each
    pilot has the most up-to-date version of the checklist.

  \item Radio frequency and channel cards.

  \item Coded authentication cards.

  \item Appropriate airfield diagrams.

  \item Appropriate weapon checklists.

  \item AO status and known threat information.

  \item Emergency checklists.

  \item Minimum safe altitudes for all traversed areas must be briefed for each
    phase of flight.

\end{subsectionenumerate}

