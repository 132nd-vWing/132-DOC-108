\section{ACM}
\label{sec:ACM}

Air Combat Manouvers or WVR dogfighting, or Section Engaged Manouvering are
synonyms for maneuvering within visual range against another fighter aircraft
in order to destroy it as quickly as possible.

For the purposes of this SOP there are very few Tomcat specific SOP's to
consider, as the general art is detailed in TTP A-A 1 and P-1289 from the US
Navy. Rather than duplicate any documentation there, here are the SOP items to
note

\subsection{References}


\textbf{\underline{\href
 {https://www.cnatra.navy.mil/local/docs/pat-pubs/P-1289.pdf}
 {P-1289 CNATRA - Section Engaged Manouver}}}


\narrow{
  \begin{enumerate}[
    align=parleft,
    itemsep=-0.25em,
    labelsep=1.0cm,
    leftmargin=*
  ]

    \item[\textbullet~Ch2] BFM TACADMIN
    \item[\textbullet~Ch4] BASIC AERODYNAMIC REVIEW
    \item[\textbullet~Ch5] OFFENSIVE BFM
    \item[\textbullet~Ch6] DEFENSIVE BFM
    \item[\textbullet~Ch7] HIGH ASPECT BFM
    \item[\textbullet~Ch8] SECTION ENGAGED MANOUVERING (skim)

  \end{enumerate}
}

\textbf{\underline{\href
 {https://cloud.132virtualwing.org/s/DdLFQX4LLTExij2}
 {132-TTP-Air-to-Air-1 v1.0}}}

Note: This SOP supersedes A-A TTP based on the different and more specific
Tomcat detail.

\subsection{Communicate}

The specific part of ACM that stands out for the Tomcat is the feature of two
pairs of eyes. This is a key advantage and it should be used. When reading the
P-1289 Ch 8 above we can see how the Wingman works in a flight of two and it is
reasonable to extend that extra pair of eyes into the fight.

Once the fight has transferred to the visual arena, defined by the Pilot, with
the call "Tally", the RIO and his radar have little use. Instead, the RIO
should concentrate on these helpful factors.

In 1 vs 1 engagements the RIO should assist in keeping tally and use
descriptive commentary on the ICS to provide the pilot with feedback on:

\begin{itemize}
 \item Speed, especially cornering speed or excessive or too low speed
 \item Fuel, in case of shortage and Bingo states
 \item Position of the Bandit
 \item Bandit prediction
 \item Training items
\end{itemize}

For SA calls, simple Clockface calls with high, low or co alt calls make sense.
If the bandit seems to be gaining angles, stating his aspect or any other
pertinent information is useful.

The RIO should also be sanitizing, and if briefed in charge of Flares or other
CM's as the Pilot requested.

\newpage

\subsection{Aviate}

\subsubsection*{Fuel}

Jettisoning Fuel tanks is an emergency procedure as it is generally not needed,
however in life or death situations, jettisoning them should be called out
early by the Pilot for the RIO to Perform safely.

\subsubsection*{Stores}

Jettisoning stores including Phoenix's is an emergency procedure coming under
the same advice above.

\subsubsection*{Roll SAS to off}

The Roll SAS should be disabled before hard maneuvering in order for the SAS
not to counter High units of Angle of Attack (over 17) and create roll reversal
situations

\subsubsection*{Gun rate}

Guns should be set to high rate of fire

\subsubsection*{ACM cover}

ACM cover is to specifically allow Phoenix's and AIM7's to leave the rails much
sooner and could be flipped unless a radar lock is found at short range for the
Fox2. ACM cover forces both Phoenix and AIM7 into BORESITE mode on the ACM
panel and shines the beam ahead for Fox1. The usage depends on tally and range
and missile selection (fleeing bandit may need a Fox1 or even Phoenix).

\subsection{Gameplans}

Gameplans should focus on
{
 \setlength\parskip{0em}
 \begin{enumerate}[label=\alph*)]

   \item Not accepting merges (Aborting they can outrun most planes, BVR
     Tomcats should be winning with a skate)

   \item Energy fights in the vertical. The Tomcat has mediocre 12-13 degrees
     turn rate and won't win versus many modern fighters in any flat fight, but
     its advantages are in acceleration, both vertically in Afterburner, and
     unloading and extending to escape. They may also win a fuel fight if
     stagnating a fight long enough to concern a short range fighter.

 \end{enumerate}
}

For tactics and procedures, consult the relevant parts of TTP A-A.
