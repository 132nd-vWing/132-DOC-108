\section{ACM}
\label{sec:ACM}

Air Combat Manoeuvring or WVR dogfighting, or Section Engaged Manoeuvring are
synonyms for manoeuvring within visual range against another fighter aircraft
in order to destroy it as quickly as possible.

For the purposes of this SOP, there are very few Tomcat specific items to
consider, as the general art is detailed in \href
  {https://cloud.132virtualwing.org/s/DdLFQX4LLTExij2}
  {132\nd Air to Air TTP} and P-1289 from the US Navy. Rather than duplicating
  any documentation, here are some SOP items to note:

\subsection{References}


\textbf{\underline{\href
 {https://www.cnatra.navy.mil/local/docs/pat-pubs/P-1289.pdf}
 {P-1289 CNATRA - Section Engaged Manouver}}}


\narrow{
  \begin{enumerate}[
    align=parleft,
    itemsep=-0.25em,
    labelsep=1.0cm,
    leftmargin=*
  ]

    \item[\textbullet~Ch2] BFM TACADMIN
    \item[\textbullet~Ch4] BASIC AERODYNAMIC REVIEW
    \item[\textbullet~Ch5] OFFENSIVE BFM
    \item[\textbullet~Ch6] DEFENSIVE BFM
    \item[\textbullet~Ch7] HIGH ASPECT BFM
    \item[\textbullet~Ch8] SECTION ENGAGED MANOUVERING (skim)

  \end{enumerate}
}

\textbf{\underline{\href
 {https://cloud.132virtualwing.org/s/DdLFQX4LLTExij2}
 {132-TTP-Air-to-Air-1 v1.0}}}

  Note: This SOP supersedes A-A TTP based on the different and more specific
Tomcat detail.

\subsection{Communicate}

The specific part of ACM that stands out for the Tomcat is the feature of two
pairs of eyes. This is a key advantage and it should be used. When reading the
P-1289 Ch 8. above, we can see how the Wingman works in a flight of two and it
is reasonable to extend that extra pair of eyes into the fight.

Once the fight has transferred to the visual arena, defined by the Pilot, with
the call "Tally", the RIO's capabilities to find and lock targets with all his
BVR-specialized radar functions are very limited.

In 1 vs. 1 engagements the RIO should assist in keeping Tally on Bandits and
Visual on Friendlies with descriptive commentary on the ICS to provide the
pilot with feedback on:

\begin{itemize}
 \item Speed, especially cornering speed or excessive or too low speed
 \item Altitude, when fighting in the vertical or once the fight is being taken
   towards the deck
 \item Fuel, in case of shortage and Bingo states
 \item Position of the Bandit
 \item Bandit prediction
\end{itemize}

The most important calls the RIO should make are to inform about predicting or
observing an offensive Bandit pulling lead to employ SRMs or guns - plus actual
missile launches and gun shots - so that the pilot can react with evasive
manoeuvres.

For SA calls, simple clock-face calls along with high, low or co-alt readouts
are appropriate. If the bandit seems to be gaining angles, stating their aspect
or any other pertinent information is useful.

The RIO should also be sanitizing, and remain in charge of Flares or other
counter measures as the Pilot requested.

\newpage

\subsection{Aviate}

\subsubsection*{Fuel}

Jettisoning fuel tanks is an emergency procedure as it is generally not needed.
However in life or death situations, jettisoning them should be called out
early by the Pilot for the RIO to perform safely.

\subsubsection*{Stores}

Jettisoning stores, including Phoenixes is an emergency procedure, warranting
the same considerations as for jettisoning fuel (see above).

\subsubsection*{Roll SAS to off}

The Roll SAS should be disabled before hard manoeuvring in order for the SAS
not to counter high Angle of Attack (over 17 units) and create roll reversal
situations.

\subsubsection*{Gun rate}

GUN RATE switch should be set to HIGH for aerial gunnery.

AIR SOURCE has to be set to L ENG, R ENG or BOTH - otherwise the lack of
internal bleed air pressure will prevent the gun from firing.

\subsubsection*{ACM Cover}

The ACM cover has multiple functions. Firstly it allows access to the ACM
jettison button, which will jettison of stores selected on the RIO’s ARMAMENT
panel. More importantly however, it impacts the WCS behaviour by enforcing
BRSIT (boresight) mode, SW COOL on, MSL PREP on, and sets the gunrate to high.

For the AIM-54, ACM Active causes the missile's seeker to be positioned toward
the current WCS track (if available), commanded active before launch, and
reduces the launch to eject from 3 seconds to 1 - making this an effective fire
and forget weapon at shorter ranges.

The AIM-7 will be affected by the enforced boresight mode, using the continuous
wave flood antenna to paint an area which causes the missile to track the
strongest return regardless of any single target track that may be selected and
disabling any pulse doppler based guidance.

One must remain aware of the location of your wingman / lead when utilising
this mode as you retain less control over the missile's behaviour.

\subsection{Gameplans}

Gameplans should focus on
{
 \setlength\parskip{0em}
 \begin{enumerate}[label=\alph*)]

   \item Not accepting merges. Tomcats are able to outrun most adversaries,
    and should focus on remaining in BVR by employing Skate and Recommit
    tactics.

   \item Energy fights in the vertical. The Tomcat has mediocre 12-13 degrees
     turn rate and won't win versus many modern fighters in any flat fight, but
     its advantages lie in acceleration, both vertically in Afterburner, and
     horizontally by unloading and extending to escape. They may also win a
     fuel fight if stagnating a fight long enough to concern a short range
     fighter.

 \end{enumerate}
}

For tactics and procedures, consult the relevant parts of TTP A-A.
