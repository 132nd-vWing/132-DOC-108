\section{AIR TO GROUND}

\subsection{The Tomcat Mission}

The Tomcat is able to employ a variety of ground ordnance ranging from dumb
(free-fall) bombs, to precision guided munitions (PGM) with Laser-Guided
Bomb Units (GBU), and able to assisting in suppression of enemy air defence
(SEAD) missions with Tactical Air Launch Decoys (TALDS) to open a window for
the destruction of enemy air defences (DEAD) missions.

The addition of Air to Ground in the early 90s (16 years after it's
introduction) brought several advantages to the platform:

\begin{itemize}

  \item Carry mixed loads and is thus able to protect itself.

  \item Out-range and out-speed many other airframes.

  \item Carry large bombs that smaller fighters are unable to .

\end{itemize}

However, these have several limitations, including:

\begin{itemize}

  \item Dumb bomb deliveries restricted to visual delivery only with DL and INS
    unable to trusted.

  \item Bomb delivery options are low level, multiple dive toss
    (inaccurate) and medium-high level dive bombing.

\end{itemize}

\subsection{LANTIRN}

The LANTIRN can be carried at all altitudes when not powered. However, it
restricts the operational altitude to a Density Altitude of max. 25,000 ft once
powered - due to the possibility of the Laser Range Finder Receiver catching
fire in thinner air

The LANTIRN has poor resolution at distance and is best used for infrastructure
targeting and long-range interdiction at medium altitude, level bombing.

\subsection{CAS}

CAS is possible, though remains limited and not part of our main mission set.

The Pilot is always in command of the CAS flow and needs to ensure he, the RIO
and the JTAC/FAC are aware of the platforms capabilities.

A CAS qualified RIO may assist the pilot by talking to the JTAC/FAC up until
the "Tipping in" call by the pilot, after which the pilot continues the radio
communication with the JTAC/FAC.

CAS is restricted to talk-on target correlation, with no capability for Bomb on
Coordinate (BOC) and so is limited to clear weather operation.

\newpage

\subsection{Gun}

The on board 20mm gun is accurate against soft targets and un-armoured vehicles
but it's upward-angled barrel axis requires steep dives and only high-angle
strafing (HAS) is recommended.

AIR SOURCE must be set to \textbf{L ENG, R ENG} or \textbf{BOTH} - otherwise
the lack of internal bleed air pressure will prevent the gun from firing.

The Weapon Selector Hat on the pilot's control stick has to be in the
\textbf{GUN} position, the (HUD) MODE A/G button has to be pushed in and the
RIO's WCS switch has to be on \textbf{XMIT} for the gun pipper to be precise in
'CCIP' by radar ground ranging.

\subsection{Other Air to Ground Employment}

The Weapon Selector Hat on the pilot's control stick has to be in the
\textbf{OFF} position for all other A/G weapon employments, otherwise the Bomb
Release Button on the stick will not function and some HUD indications like the
bomb fall line will not be displayed.

\subsection{SOPs}

All further SOPs are contained in the relevant A-G TTPs available in the
documents section of the 132\nd website as there is little airframe specific
restrictions.
