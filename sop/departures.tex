\section{DEPARTURES}

\subsection{Airfield}

\subsubsection{Taxi}

\boxed{
  Wings will remain Over-Swept when taxiing.

  Antiskid should be set to OFF during taxi and speed kept beneath 15kts
  (NATOPS)
}

Taxi Interval/Speed. Minimum taxi interval is 100m (\textapprox300 feet) in
trail. This spacing should be reduced when holding short of entering the
runway. Speed should be limited to \textbf{15 knots and 10 knots in turns}.
Increase separation if server instability occurs or warn others of personal
lag.  Before taxi can commence, each flight member must be aware of the
location of all the aircraft in the flight. Taxi starts when the last wingman
responds to the taxi directive.

\subsubsection{Hold short}

At 'hold short':

\begin{itemize}
  \item Flight should be stationary

  \item Wings will be returned to \textbf{Auto} (and extension checked) once
    clear of obstruction

  \item Anti-Skid should be set to \textbf{BOTH}

  \item Flaps will be deployed with verbal cue on the flight channel:

  \begin{itemize}

    \item \textbf{Full flaps} if heavy load and fuel is not tight

    \item \textbf{Manoeuvre} flaps elsewise

  \end{itemize}

  \item Beacon will be set to \textbf{ON (cuts the flashing Nav lights)}

  \item Remaining checklist items such as Trim set for T/O
\end{itemize}

\subsubsection{Line Up}

The following options are acceptable and will be briefed or stated prior to
taking the active:

Formation (no sidewind)

\begin{itemize}

  \item Two ship lineup in formation for a \textbf{formation departure} in
    echelon as a two-ship only

  \item Four ship lineup in formation for a \textbf{formation departure} with
    20 second separations on the element (Wake Turbulence)

\end{itemize}

Individual

\begin{itemize}

  \item Two ship lineup in formation for \textbf{individual departure} with 20
    seconds spacing (weather/wind)

  \item Four ship lineup in formation for \textbf{individual departure} with 20
    second separations individually

\end{itemize}

Flight Line-up: Lead will take the downwind side in a crosswind or the opposite
from the taxiway they entered from. If synchronized Formation approved, two
will stagger. \textbf{20 seconds minimum} between elements. Before entering the
runway, lead needs to specify if this is a formation departure or of not, what
spacing in seconds should be used.

\newpage

\subsubsection{Take off (land-based)}

All take-offs will be military power only.

For Carrier Ops, please see \href
  {https://cloud.132virtualwing.org/s/wSMBb66EEEJztNW}
  {132\nd TTP-19 Carrier Operations}

Both

\begin{itemize}

  \item After line-up is finished, each member of the Flight will call with
    their flight number: "1 lined up, on the brakes".

  \item After all elements of the Flight are on the brakes, Lead calls "run'em
    up" to spool engines to 80% with brakes on.

  \item Each flight member including lead should in order: (a) check all
    instruments and warning lights indicate correct systems operation; (b) if
    all criteria are met then respond with "2 in the green".

\end{itemize}

Individual

\begin{itemize}

  \item When the last flight member responds, Flight Lead will call "1
    releasing!"

  \item Remainder call out "2, releasing", in 20 second intervals

  \item Flight members call "Airborne" as it happens

  \item Seek visual contact with all Flight members during the rejoin

\end{itemize}

Formation

\begin{itemize}

  \item If a formation take off is performed, the brevity will be in a regular
    cadence: "\textit{Brakes, Brakes, Brakes, Buster!}", followed by
    "\textit{Rotate!}".

  \item An element (3 \& 4) will wait, and the Element Leader will call the
    same on Forward radio at the 20 seconds.

  \item When each flight member has taken off, raised the landing gear, and
    visually acquired the aircraft ahead of him then they should confirm these
    three things with the call "Airborne, gear up, flaps up, visual". This
    indicates to the lead pilot that all aircraft are airborne and proceeding
    to initial rejoin.

\end{itemize}

\subsubsection{Climb out and rejoin}

FL should ease up to the contract climb speed without using all available
throttle. AoA 5 units is the best efficiency for climbout (see fuel economy).

\begin{itemize}

  \item Retract Taxi/Landing Lights (They auto retract but the button does not)

  \item Check that Manoeuvre flaps are operating and retracting by AoA.

  \item Check TACAN sweet and range

  \item TACAN yardstick will help re-joins, as will remaining radar "tied", or
    just using Datalink.

\end{itemize}

Flight members should report yardstick when not visual and the flight should
operate as if a "Blind" call was made with regular callouts.

For IMC conditions, or otherwise delayed take-offs, the Mission Commander
should pre brief a Rejoin RV point. Cloud penetration should be managed
efficiently and with deconfliction in mind.
