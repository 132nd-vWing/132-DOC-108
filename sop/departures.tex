\section{DEPARTURES}

\subsection{Airfield}

\subsubsection{Taxi}

\begin{itemize}

  \item Each flight member must be aware of the location of all the aircraft in
    the flight.

  \item Wings will remain Over-Swept when taxiing.

  \item Antiskid should be set to \textbf{OFF}

  \item Speed must not exceed \textbf{15 knots} or \textbf{10 knots in turns}.

  \item Minimum taxi spacing is \textbf{100m} (\textapprox300 feet) in direct
  trail

  \begin{itemize}
    \item Reduce separation when holding

    \item Increase separation should you encounter server instability or
      individual lag.
  \end{itemize}

\end{itemize}

\subsubsection{Holding Point}

\begin{itemize}

  \item Flight should be stationary

  \item Wings will be reconnected and reset to 20\textdegree/\textbf{Auto} once
    clear of obstruction and verified.

  \item Flaps will deployed with verbal cue on the flight channel:

  \begin{itemize}

    \item \textbf{Full flaps} if heavy load and fuel is not tight

    \item \textbf{Manoeuvre} flaps otherwise

  \end{itemize}

  \item Remaining checklist items such as Trim set for T/O

\end{itemize}

\subsubsection{Lining Up}

The following options are acceptable and will be briefed or stated prior to
lining up

\begin{itemize}

  \item Two-ship: line up in \textbf{echelon formation} with flight lead taking
    the downwind with a crosswind, or the opposite side from the taxiway they
    entered from.

  \item Four-ship: line up in two Two-ship formations in series.

  \item Anti-Collision lights to \textbf{ON} upon crossing the runway threshold.

  \item Once lined up, set Anti-Skid to \textbf{BOTH} and maintain foot brakes.

\end{itemize}

Formation Departure (no crosswind)

\begin{itemize}

  \item Four-Ship: Minimum of 20 second break-release intervals between
    elements to avoid wake turbulence

\end{itemize}

Individual Departure

\begin{itemize}
  \item Each aircraft \textbf{departs individually} with a minimum of 20
    seconds brake-release intervals to avoid wake turbulence or crosswind drift.
\end{itemize}

If performing a (closed) formation departure, wingmen / elements will stagger.
Before entering the runway, lead needs to specify if this is a formation
departure and if not, what break-release interval should be used.

\newpage

\subsubsection{Take off (land-based)}

Take-offs in F110-powered Tomcats (F-14A+/B/D models) are restricted to full
military thrust / dry power. The use of afterburner is for emergency or
non-standard wartime operations only.

For Carrier operations, please see \href
  {https://cloud.132virtualwing.org/s/wSMBb66EEEJztNW}
  {132\nd TTP-19 Carrier Operations}

All radio calls are expected to take place on Internal, unless otherwise
stated.

\begin{itemize}

  \item Each member in the flight will call with their flight number: "1 lined
    up, on the brakes".

  \item Once all aircraft have reported on the brakes, Lead calls
    "run 'em up" to spool engines to 80\%

  \item Each flight member including lead should: check all instruments and
    warning lights to confirm correct systems operation and if all criteria are
    met, report with "\textit{<flight>} in the green".

\end{itemize}

Individual

\begin{itemize}

  \item When the last flight member reports in the green, Lead will call "1
    releasing!"

  \item Each aircraft departs in sequence following the pre-briefed
    brake-release interval and reports "\textit{<flight>}, releasing"

  \item Report "\textit{<flight>}, airborne" as it happens

  \item Seek visual contact with all Flight members during the rejoin

\end{itemize}

Formation

\begin{itemize}

  \item If a formation take off is performed, the phraseology will be in a
    regular cadence: "\textit{Brakes, Brakes, Brakes, Buster!}", followed by
    "\textit{Rotate!}".

  \item An element waits the pre-briefed break-release interval,
    and calls the same for the element.

  \item Element leaders should reduce power slightly to 98\% a few seconds
    after brake release if there is enough available runway length, aircraft
    weight \& performance, and the atmospheric conditions to do so with
    sufficient safety margin to provide wingman manoeuvrability.

  \item When each flight member is airborne, raised the landing gear, and
    visually acquired the aircraft ahead of him then they should confirm these
    three things with the call "Airborne, gear up, flaps up, visual". This
    indicates to the lead pilot that all aircraft are airborne and proceeding
    to initial rejoin.

\end{itemize}

Formation departures are not permitted under the following conditions:

\begin{itemize}
  \item There is a crosswind component exceeding 15 kts
  \item The runway is wet or slippery
  \item The runway width is < 125 feet
  \item The weather has a ceiling lower than 500 feet AGL
  \item Asymmetric stores loadouts
  \item Unsafe loadouts like hung ordnance
\end{itemize}

\newpage

\subsubsection{Climb out and rejoin}

The Tomcats' optimum climb rate is marked on the AoA indicator at 5 units. For
a 4/2/2 20,0 loaded Tomcat (~70,000 lbs) this is around 340-350 KCAS. If the
current airspace (CTR / TMA) has speed restrictions, Lead should request
clearance for "no speed restrictions" from ATC.

\begin{itemize}

  \item Check that Manoeuvre flaps are operating and retracting by AoA.

  \item Check TACAN sweet and range

  \item TACAN yardstick will help re-joins, as will remaining radar-trail, or
    just using Datalink.

\end{itemize}

Flight members should report yardstick when not visual and the flight should
operate as if a "Blind" call was made with regular callouts.

In situations where Radar-Trail and Yardstick are not feasible (e.g. delayed
individual departures)," the Mission Commander should pre brief a Rejoin RV
point. Cloud penetration should be managed efficiently and with deconfliction
in mind.
